\documentclass[mstat,12pt]{unswthesis}

\usepackage{color}
\usepackage{fancyvrb}
\newcommand{\VerbBar}{|}
\newcommand{\VERB}{\Verb[commandchars=\\\{\}]}
\DefineVerbatimEnvironment{Highlighting}{Verbatim}{commandchars=\\\{\}}
% Add ',fontsize=\small' for more characters per line
\usepackage{framed}
\definecolor{shadecolor}{RGB}{248,248,248}
\newenvironment{Shaded}{\begin{snugshade}}{\end{snugshade}}
\newcommand{\AlertTok}[1]{\textcolor[rgb]{0.94,0.16,0.16}{#1}}
\newcommand{\AnnotationTok}[1]{\textcolor[rgb]{0.56,0.35,0.01}{\textbf{\textit{#1}}}}
\newcommand{\AttributeTok}[1]{\textcolor[rgb]{0.77,0.63,0.00}{#1}}
\newcommand{\BaseNTok}[1]{\textcolor[rgb]{0.00,0.00,0.81}{#1}}
\newcommand{\BuiltInTok}[1]{#1}
\newcommand{\CharTok}[1]{\textcolor[rgb]{0.31,0.60,0.02}{#1}}
\newcommand{\CommentTok}[1]{\textcolor[rgb]{0.56,0.35,0.01}{\textit{#1}}}
\newcommand{\CommentVarTok}[1]{\textcolor[rgb]{0.56,0.35,0.01}{\textbf{\textit{#1}}}}
\newcommand{\ConstantTok}[1]{\textcolor[rgb]{0.00,0.00,0.00}{#1}}
\newcommand{\ControlFlowTok}[1]{\textcolor[rgb]{0.13,0.29,0.53}{\textbf{#1}}}
\newcommand{\DataTypeTok}[1]{\textcolor[rgb]{0.13,0.29,0.53}{#1}}
\newcommand{\DecValTok}[1]{\textcolor[rgb]{0.00,0.00,0.81}{#1}}
\newcommand{\DocumentationTok}[1]{\textcolor[rgb]{0.56,0.35,0.01}{\textbf{\textit{#1}}}}
\newcommand{\ErrorTok}[1]{\textcolor[rgb]{0.64,0.00,0.00}{\textbf{#1}}}
\newcommand{\ExtensionTok}[1]{#1}
\newcommand{\FloatTok}[1]{\textcolor[rgb]{0.00,0.00,0.81}{#1}}
\newcommand{\FunctionTok}[1]{\textcolor[rgb]{0.00,0.00,0.00}{#1}}
\newcommand{\ImportTok}[1]{#1}
\newcommand{\InformationTok}[1]{\textcolor[rgb]{0.56,0.35,0.01}{\textbf{\textit{#1}}}}
\newcommand{\KeywordTok}[1]{\textcolor[rgb]{0.13,0.29,0.53}{\textbf{#1}}}
\newcommand{\NormalTok}[1]{#1}
\newcommand{\OperatorTok}[1]{\textcolor[rgb]{0.81,0.36,0.00}{\textbf{#1}}}
\newcommand{\OtherTok}[1]{\textcolor[rgb]{0.56,0.35,0.01}{#1}}
\newcommand{\PreprocessorTok}[1]{\textcolor[rgb]{0.56,0.35,0.01}{\textit{#1}}}
\newcommand{\RegionMarkerTok}[1]{#1}
\newcommand{\SpecialCharTok}[1]{\textcolor[rgb]{0.00,0.00,0.00}{#1}}
\newcommand{\SpecialStringTok}[1]{\textcolor[rgb]{0.31,0.60,0.02}{#1}}
\newcommand{\StringTok}[1]{\textcolor[rgb]{0.31,0.60,0.02}{#1}}
\newcommand{\VariableTok}[1]{\textcolor[rgb]{0.00,0.00,0.00}{#1}}
\newcommand{\VerbatimStringTok}[1]{\textcolor[rgb]{0.31,0.60,0.02}{#1}}
\newcommand{\WarningTok}[1]{\textcolor[rgb]{0.56,0.35,0.01}{\textbf{\textit{#1}}}}


%%%%%%%%%%%%%%%%%%%%%%%%%%%%%%%%%%%%%%%%%%%%%%%%%%%%%%%%%%%%%%%%%%
% 
% OK...Now we get to some actual input.  The first part sets up
% the title etc that will appear on the front page
%
%%%%%%%%%%%%%%%%%%%%%%%%%%%%%%%%%%%%%%%%%%%%%%%%%%%%%%%%%%%%%%%%%

\title{Rapport de groupe en Sciences des Données 4  }

\authornameonly{Victor Hugo, Albert Enstein. }


\author{\Authornameonly}

\copyrightfalse
\figurespagefalse
\tablespagefalse

%%%%%%%%%%%%%%%%%%%%%%%%%%%%%%%%%%%%%%%%%%%%%%%%%%%%%%%%%%%%%%%%%
%
%  And now the document begins
%  The \beforepreface and \afterpreface commands puts the
%  contents page etc in
%
%%%%%%%%%%%%%%%%%%%%%%%%%%%%%%%%%%%%%%%%%%%%%%%%%%%%%%%%%%%%%%%%%%


\input{header.tex}

\renewcommand{\contentsname}{Table des matières}

\renewcommand{\chaptername}{Chapitre}



\begin{document}

\beforepreface

%\afterpage{\blankpage}

% plagiarism

\prefacesection{Déclaration de non plagiat}

\vskip 2pc \noindent Nous déclarons que ce rapport est le fruit de notre seul travail, à part lorsque cela est indiqué  explicitement. 

\vskip 2pc  \noindent Nous acceptons que la personne évaluant ce rapport puisse, pour les besoins de cette évaluation:
\begin{itemize}
\item la reproduire et en fournir une copie à un autre membre de l'université; et/ou,
\item en communiquer une copie à un service en ligne de détection de plagiat (qui pourra en retenir une copie pour les besoins d'évaluation future).
\end{itemize}

\vskip 2pc \noindent Nous certifions que nous avons lu et compris les règles ci-dessus.\vspace{24pt}

\vskip 2pc \noindent En signant cette déclaration, nous acceptons ce qui précède.
\vskip 2pc \noindent
Signature: \rule{7cm}{0.25pt} \hfill Date: \rule{4cm}{0.25pt} \\[1cm]
Signature: \rule{7cm}{0.25pt} \hfill Date: \rule{4cm}{0.25pt} \\[1cm]
\vskip 1pc

%\afterpage{\blankpage}

% Acknowledgements are optional


\prefacesection{Remerciements}

{\bigskip}Nos plus sincères remerciements vont à nos encadrants
pédagogiques pour les conseils avisés sur notre travail.\\[1cm] Nous
remercions aussi \ldots{}\\[1cm] 

{\bigskip\bigskip\bigskip\noindent} 

%\afterpage{\blankpage}

% Abstract

\prefacesection{Résumé}

L'image ci-dessous vous donne quelques conseils pour rédiger un bon
résumé.

\par

\bigskip \includegraphics[width=10cm,height=10cm]{img/good-abstract.png}

%\afterpage{\blankpage}


\afterpreface





%%%%%%%%%%%%%%%%%%%%%%%%%%%%%%%%%%%%%%%%%%%%%%%%%%%%%%%%%%%%%%%%%%
%
% Now we can start on the first chapter
% Within chapters we have sections, subsections and so forth
%
%%%%%%%%%%%%%%%%%%%%%%%%%%%%%%%%%%%%%%%%%%%%%%%%%%%%%%%%%%%%%%%%%%



%%%%%%%%%%%%%%%%%%%%%%%%%%%%%%%%%%%%%

%\afterpage{\blankpage}


\hypertarget{introduction}{%
\chapter{Introduction}\label{introduction}}

Vous devez introduire votre projet qui a sans doute évolué par rapport à la version de décembre. 
Quel est le nouveau périmètre du projet si celui-ci a évolué ?

A qui s'adresse t'il ? 

En quoi votre projet est original et se démarque des sites déjà existants ? Comment pourriez vous en faire la publicité ?

Conclure l’introduction par l'url vers la vidéo de démonstration de votre projet ainsi que la vidéo de soutenance.

Donner le plan du mémoire.

%%%%%%%%%%%%%%%%%%%%%%%%%%%%%%%%%%%%%
\hypertarget{Gestion-de-projet}{%
\chapter{Gestion de projet }\label{chapGestionProjet}}

\hypertarget{IntroductionGestionProjet}{%
\section{Introduction}\label{IntroGestionProjet}}

Vous introduirez ce chapitre et préciserez la méthodologie suivie. Vous donnerez le lien vers le site GitHub.

Donner le plan de la section.

\hypertarget{collaborationGit}{%
\section{Collaboration avec Git}\label{collaborationGit}}
Vous décrirez comment vous vous êtes servis de Git pour le développement collaboratif. Avez-vous utilisé des branches ? Comment avez-vous maîtrisé les conflits de merge ? Comment avez-vous alimenté votre fichier .gitignore et pourquoi ? Avez-vous profité de l'historique de commits (\texttt{git log}) pour récupérer du code à partir d'une ancienne version de votre projet ?  etc. 

\hypertarget{contributionParEtudiant}{%
\section{Contribution}\label{contributionParEtudiant}}

Vous complèterez 2 tableaux :
\begin{itemize}
\item un tableau avec la contribution de chacun par tache est obligatoire : par exemple tache 1 : base de données réalisée par l’étudiant X à 30\% et Y à 70\%.
\item un tableau avec la contribution de chacun par page est obligatoire : par exemple Page d’accueil  :  réalisée par l’étudiant X à 30\% et Y à 70\%.
\end{itemize} 

Vous expliciterez les différence observable entre la contribution affichée et ce qui est visible sur Github. Commenter le nombre de commit par utilisateur, les dates…


\hypertarget{planning}{%
\section{Planning}\label{Planning}}

Le planning de janvier n’est surement pas celui qui a été suivi ce semestre. Indiquer les deux (l’ancien et le planning mis à jour).

Il faut donc expliquer pourquoi certaines taches ont été plus rapides ou lentes que prévu.  Comment vous vous êtes adaptés ? 



\hypertarget{ConclusionGestionProjet}{%
\section{Conclusions}\label{ConcluGestionProjet}}



%%%%%%%%%%%%%%%%%%%%%%%%%%%%%%%%%%%%%

\hypertarget{BD}{%
\chapter{Base de données}\label{chapBD}}


\hypertarget{IntroductionBD}{%
\section{Introduction}\label{IntroBD}}

Vous introduirez ici le chapitre et donnerez le ou les url des sites sur lesquels vous avez récupéré des données.
 
  Dans les jeux de données récupérés, expliquer pourquoi vous avez
  sélectionné certaines tables. Vous devez indiquer
  comment les données sont stockées (le format), la taille des fichiers,
  le nombre, \ldots{}
  
  Donner le plan de la section.


\hypertarget{descriptifTraitementBD}{%
\section{Traitements réalisés}\label{traitementBD}}

Vous décrirez ici les étapes réalisées pour pré-traiter vos données. Expliquer quels choix vous avez réalisés pour filtrer les lignes et
  les colonnes (éventuellement réduire le périmètre du projet) et
  décrire les critères de sélection (e.g., ne garder que 5 colonnes sur
  les 15, ne garder que les lignes qui correspondent à une ville en
  particulier, \ldots).

 Préciser les nettoyages réalisés avant l'import comme l'uniformisation
  des valeurs des champs (\emph{e.g.}, Mr, M., Monsieur, \ldots) ou le
  remplissage des valeurs manquantes par une valeur moyenne \ldots{}

\hypertarget{descriptif-des-tables}{%
\section{Descriptif des tables}\label{descriptif-des-tables}}

Pour chaque table conservée, préciser le nombre de lignes et de
  colonnes après filtrage, lister les colonnes et donner pour chacune le
  type, la signification du champ et des caractéristiques (unique, clés,
  valeur manquante, \ldots) en remplissant le tableau ci-dessous.

\begin{longtable}[]{@{}cccc@{}}
\caption{Nom de la table (nombre de lignes \(\times\) nombre de
colonnes)}\tabularnewline
\toprule
Nom colonne & Type & Signification & Caractéristique \\
\midrule
\endfirsthead
\toprule
Nom colonne & Type & Signification & Caractéristique \\
\midrule
\endhead
& & & \\
\bottomrule
\end{longtable}

\hypertarget{MCD}{%
\section{Modèle conceptuel des données}\label{mcd}}

Pour le MCD, inclure une image réalisée avec le logiciel Mocodo
  {[}\url{https://www.mocodo.net/}{]} telle que celle visible sur la
  Figure\(~\)\ref{uml} ci-dessous :

  \begin{figure}
  \hypertarget{uml}{%
  \centering
  \includegraphics[width=8cm,height=4cm]{img/uml.png}
  \caption{Relations.}\label{uml}
  }
  \end{figure}
  
\hypertarget{MOD}{%
\section{Modèle organisationnel des données}\label{mod}}

  Pour le MOD, inclure une image réalisée avec le designer de phpmyadmin


\hypertarget{requuxeates-ruxe9alisuxe9es}{%
\section{Requêtes réalisées}\label{requuxeates-ruxe9alisuxe9es}}

Pour les requêtes importantes dans votre interface, les exprimer en langage naturel puis en SQL. Puis
donner le résultat obtenu (ou un extrait) et expliquer ce résultat.


\hypertarget{concluBD}{%
\section{Conclusions}\label{concluBD}}

Préciser la durée de la mise en place de la BD et les  principales difficultés rencontrées.

\normalsize


%%%%%%%%%%%%%%%%%%%%%%%%%%%%%%%%%%%%%

\hypertarget{description}{%
\chapter{Description technique et fonctionnelle}\label{description}}

\hypertarget{introDescription}{%
\section{Introduction}\label{introDescription}}


Introduire le chapitre et donner le plan de la section.


\hypertarget{Description technique}{%
\section{Description technique}\label{DescriptionT}}

Vous listerez les  technologies utilisées (Git, Php, Javascript, Mamp…) 

Vous inclurez un schéma de l’architecture de votre application


\hypertarget{Description fonctionnelle}{%
\section{Description fonctionnelle}\label{DescriptionF}}


 Par écran, 
 \begin{itemize}
 \item quelles sont les fonctionnalités, que peut faire l’utilisateur ? 
 \item quelle gestion des erreurs avez vous prévu : par exemple, sur un écran de connexion, on peut vérifier que un email contient un ‘@‘, qu’un mot de passe n’est pas trop facile..
 \end{itemize}

Ne pas hésiter à lister les fonctionnalités que vous n’avez pas eu le temps d’implémenter pour montrer que vous y avez quand même pensé. Idem pour la gestion des erreurs.

\hypertarget{concluBD}{%
\section{Conclusions}\label{concluDescription}}

Conclure et lister les difficultés rencontrées.



\hypertarget{conclusion-et-perspectives}{%
\chapter{Conclusions et perspectives}\label{conclusion-et-perspectives}}




Résumer vos contributions puis donner des perspectives à vos travaux. C est pour nous la partie importante : s’il l’on vous prenait en stage pour 2 mois sur votre sujet, auriez vous des idées et seriez vous autonomes pour améliorer votre application. 
Combien de temps auriez vous besoin pour ajouter ces fonctionnalités ? 

On attend de vous deux types de perspectives : des perspectives à court
terme pour améliorer rapidement votre approche et des perspectives à
plus long terme qu'elles soient liées à la science des données ou au
domaine métier pour lequel vous avez travaillé.



\hypertarget{annexes}{%
\chapter*{Annexes}\label{annexes}}
\addcontentsline{toc}{chapter}{Annexes}



\end{document}

